% LaTex Template

\documentclass[12pt]{article}
%\usepackage{natbib}
\usepackage[letterpaper, margin=1.1in]{geometry}
\usepackage{graphicx}
\usepackage{wrapfig}
\usepackage{enumitem}
\setlist[enumerate]{itemsep=0mm}
\usepackage{multirow}
\usepackage{lscape}

\begin{document}
\noindent{Alexandra Pulwicki \\ \today}

\begin{center}
\Large \textbf{Field Report \\ May 2016}
\end{center}

Short intro
Field design 
	glaciers,
	sampling schemes (+ getting waypoints on GPS), naming system
Execution -> 
		snow depth summary table -> glacier, shape, resolution, date, waypoint range, observer order, comments
	firn corer, 
Daily synopsis
Improvement and thoughts -> camp, equipment
Potential Data Analysis -> questions (difference between people? gps uncertainity), sources of error (probe tip)


\section{Field Design}


\section{Implementation}

\subsection{Transects}
\label{sec:transects}

The transects, which include the hourglass, circle, transverse transect, and midline, were all executed in a similar way. Along each transect, waypoints were marked every 30 m. To sample these locations, a team of four people was required. The four people were roped together so that when the rope was loose between them there was approximately 10 m separating each person. The front person was responsible for navigation and waypoint marking and would follow these steps for each measurement location:
\begin{enumerate}
\item Use the GPS unit to locate each intended waypoint
\item Navigate to that location using the GPS
\item Stop and inform the team when they had arrived at the location
\item Mark a new waypoint on the GPS as the real location of the measurement (allow for auto labelling of waypoint, which was a three digit number that increased by one with subsequent waypoints). When needed, call out the waypoint label to the team.
\item In one line of a field book, write the labels `Intended' and `Real' labels for the location, as well as the easting, northing, and elevation for that location. This served as the backup for the locations, in the event of GPS failure. 
\end{enumerate}

The three people behind were all taking snow depth measurement using a graduated 3.2 m avalanche probe. Upon arriving at the waypoint they would follow these steps:
\begin{enumerate}
\item Insert the probe into the snow until the snow/ice interface was reached. Read the depth of the snow pack on the probe to 0.5 cm. Repeat two more times (total of three measurements) within a 1 m$^2$ area of the first measurement and so that the three measurements are approximately equidistant. 
\item In one line of a field book, record the `Real' waypoint value, as well as the three depth measurements. 
\end{enumerate}
Note that the interface was identified by a bright `ping' sound made by the probe and could typically be differentiated from the ice lenses, which made a dull `thud' sound in the probe. Often, layers in the snowpack could be felt with the probe. For example, the probe would move easily through low density layers such as depth hoar and would `stick' to hardback layers or ice lenses. Increasing the force applied to the probe would usually allow the probe to penetrate through hard layers. In cases where the `sticky' layer could not be penetrated, the observer would place a question mark next to the recorded depth or simply omit that measurement. Note that the probe was inserted vertically, which was not necessarily perpendicular to the snow surface.

It was originally planned for each observer to take four depth measurements in a square pattern. This was tried during the first transect and the observers found that it was difficult and considerably more time consuming to remember and record four depths. The observers found that the most efficient way to collect data was to take three depth measurements, remember the values, and then write them all down in the field book. When four measurements were taken it was too difficult to remember all the values simultaneously so the whole process would take much longer. The decision was made to decrease the number of measurements so that we could increase the number of locations measured. 

There were dedicated field books for each type of measurement rather than each observer. The first person had the `Navigation' field book, the second person had `Snow depth \# 1', the third person had `Snow depth \# 2', and the fourth person had `Snow depth \# 3'. In this way, the location of each measured value can be inferred from its location relative to the navigation person (where the location was being recorded). For example, the `Snow depth \# 3' value was located 30 m behind the waypoint location along the trajectory between the previous and current waypoint. This arrangement was preferred to having a field book for each observer because it minimized confusion and potential errors when entering and processing data.

In this arrangement, snow depth measurements could be taken every 10 m along a transect if a waypoint was marked every 30 m. For the first two transects, measurements were completed at every waypoint. However, this also proved to be too time consuming so it was decided to do measurements at every second waypoint for subsequent transects (exceptions include the midline on Glacier 4 and the lower hourglass on Glacier 2, see Table \ref{snowdepthsummary}). Waypoints that were too dangerous to access were omitted. `Bonus' waypoints were created in some instances when travelling from the last accessible waypoint to the next accessible waypoint.


INSERT DIAGRAM OF PEOPLE IN A LINE

\begin{landscape}

% Please add the following required packages to your document preamble:
% \usepackage{multirow}
\begin{table}[]
\centering
\caption{Summary information for snow depth transects}
\label{snowdepthsummary}
\begin{tabular}{ccccccl}
\textbf{Glacier}                                                             & \textbf{Shape}                                               & \textbf{Resolution}                                                      & \textbf{Date} & \textbf{\begin{tabular}[c]{@{}c@{}}GPS \\ Waypoint\\  Labels\end{tabular}} & \textbf{\begin{tabular}[c]{@{}c@{}}Observer\\  Order\end{tabular}} & \multicolumn{1}{c}{\textbf{Comments}}                                                                                                                                                                                                            \\ \hline
\multirow{7}{*}{\begin{tabular}[c]{@{}c@{}}Glacier 4\\ (G04)\end{tabular}}   & \textbf{LH}                                                  & \begin{tabular}[t]{@{}c@{}}30 m \\ (60 m for \\ upper part)\end{tabular} & 4 May 2016    & 021 -- 070                                                                 & GF--AP--CA--AC                                                     & \begin{tabular}[t]{@{}l@{}}4 depth measurement/location \\ along upper part\end{tabular}                                                                                                                                                         \\
                                                                             & \textbf{LC}                                                  & 60 m                                                                     & 6 May 2016    & 159 -- 184                                                                 & GF--AP--CA--AC                                                     &                                                                                                                                                                                                                                                  \\
                                                                             & \textbf{LM}                                                  & 90 m                                                                     & 7 May 2016    & 185 -- 207                                                                 & AP--GF--CA--AC                                                     &                                                                                                                                                                                                                                                  \\
                                                                             & \textbf{UH}                                                  & 60 m                                                                     & 5 May 2016    & 072 -- 126                                                                 & CA--GF--AP--AC                                                     &                                                                                                                                                                                                                                                  \\
                                                                             & \textbf{UC}                                                  & 60 m                                                                     & 5 May 2016    & 127 -- 157                                                                 & CA--GF--AP--AC                                                     &                                                                                                                                                                                                                                                  \\
                                                                             & \textbf{UM}                                                  & 90 m                                                                     & 7 May 2016    & 208 -- 221                                                                 & AP--GF--CA--AC                                                     & Bonus depth at WP 158 (6 May 2016)                                                                                                                                                                                                               \\
                                                                             & \textbf{UT}                                                  & 30 m                                                                     & 4 May 2016    & 004 -- 020                                                                 & GF--AP--CA--AC                                                     & 4 depth measurement/location                                                                                                                                                                                                                     \\ \hline
\multirow{7}{*}{\begin{tabular}[t]{@{}c@{}}Glacier 2 \\ (G02)\end{tabular}}  & \textbf{\begin{tabular}[t]{@{}c@{}}LH \\ \& LC\end{tabular}} & 30 m                                                                     & 11 May 2016   & 371 -- 518                                                                 & GF--AP--CA                                                         & \begin{tabular}[t]{@{}l@{}}Only two probers. Avoided crossing main \\ channel so LH \& LC were combined and \\ done together on glacier right and then \\ glacier left of the channel. Almost all \\ measurements in the dune area.\end{tabular} \\
                                                                             & \textbf{LM}                                                  & $\sim$60 m                                                               & 10 May 2016   & 355 -- 370                                                                 & AP--GF--CA--AC                                                     & \begin{tabular}[t]{@{}l@{}}Original points along supraglacial stream \\ bed so points moved to glacier right and \\ locations were approximated\end{tabular}                                                                                     \\
                                                                             & \textbf{UH}                                                  & 60 m                                                                     & 8 May 2016    & 223 -- 275                                                                 & AC--AP--CA--GF                                                     & \begin{tabular}[t]{@{}l@{}}Many corner points avoided due to \\ crevasse danger\end{tabular}                                                                                                                                                     \\
                                                                             & \textbf{UC}                                                  & 60 m                                                                     & 8 May 2016    & 276 -- 313                                                                 & AC--AP--CA--GF                                                     &                                                                                                                                                                                                                                                  \\
                                                                             & \textbf{UM}                                                  & 60 m                                                                     & 9 May 2016    & 313 -- 343                                                                 & AC--AP--CA--GF                                                     &                                                                                                                                                                                                                                                  \\
                                                                             & \textbf{UT}                                                  & 60 m                                                                     & 11 May 2016   & 519 -- 528                                                                 & GF--AP--CA                                                         & Only two probers                                                                                                                                                                                                                                 \\
                                                                             & \textbf{BT}                                                  & $\sim$60 m                                                               & 19 May 2016   & 344 -- 354                                                                 & GF--AP--CA--AC                                                     &                                                                                                                                                                                                                                                  \\ \hline
\multirow{7}{*}{\begin{tabular}[t]{@{}c@{}}Glacier 13 \\ (G13)\end{tabular}} & \textbf{LH}                                                  & 60 m                                                                     & 15 May 2016   & 745 -- 811                                                                 & AC--AP--CA--GF                                                     &                                                                                                                                                                                                                                                  \\
                                                                             & \textbf{LC}                                                  & 60 m                                                                     & 15 May 2016   & 812 -- 847                                                                 & AC--AP--CA--GF                                                     &                                                                                                                                                                                                                                                  \\
                                                                             & \textbf{LM}                                                  & 60 m                                                                     & 14 May 2016   & 714 -- 743                                                                 & AC--AP--CA--GF                                                     &                                                                                                                                                                                                                                                  \\
                                                                             & \textbf{UH}                                                  & 60 m                                                                     & 12 May 2016   & 571 -- 650                                                                 & AC--GF--CA--AP                                                     &                                                                                                                                                                                                                                                  \\
                                                                             & \textbf{UC}                                                  & 60 m                                                                     & 12 May 2016   & 529 -- 570                                                                 & AC--GF--CA--AP                                                     &                                                                                                                                                                                                                                                  \\
                                                                             & \textbf{UM}                                                  & 60 m                                                                     & 14 May 2016   & 678 -- 713                                                                 & AC--AP--CA--GF                                                     &                                                                                                                                                                                                                                                  \\
                                                                             & \textbf{UT}                                                  & 60 m                                                                     & 14 May 2016   & 660 -- 677                                                                 & AC--AP--CA--GF                                                     &                                                                                                                                                                                                                                                 
\end{tabular}
\end{table}

\end{landscape}

\subsection{Zigzag}

The zigzag sampling pattern was used to obtain many measurements within a 40 x 40 m area.  The pattern consists of two intersecting `Z' shaped transects. Snow depth is measured with random spacing (between 0.3 m and 3.0 m). 

Two teams of two people were used to complete each zigzag. The first team would navigate to the vertices of the zigzag using the GPS and place wands at each vertex. Often the tracks would not be straight between two vertices so the second team would travel between wands in as straight of a line as possible. The first person would use the avalanche probe to measure out the distance to the next measurement and then probe at that point. Probing protocol was exactly the same as for transect measurements (see Section \ref{sec:transects}). The first person would call out the depth to the second person, who was responsible for recording the distance between measurements and the depth at the measurement point. A field book was dedicated to zigzag measurements and each page would have the name of the vertex where measurements started, the distance from the previous measurement point and the depth at that point. The second person also had a sheet with uniformly distributed random numbers (generated used Matlab) and would call out these numbers in order as the distance between measurement points. The first pair of people took three SWE measurements at the predetermined location within the zigzag area once they were done navigating (see Section \ref{sec:SWE} for protocol).

 PICTURE OF ZIGZAG WITH DISTANCES
 
 \subsection{Federal Snow Sampler}
\label{sec:SWE}
 
A metric Federal Snow Sampler from Geo Scientific Ltd. was used to measure snow depth and snow water equivalent (SWE). At the predetermined locations, three measurements (within 50 cm of each other) were made using the sampler. At the snowpit locations, a total of eight measurements were made, with two measurements on each side of the snowpit. Density calculated from these values will be compared with density determined from sampling within the snow pit (see Section \ref{sec:snowpit}). 

The Federal Snow Sampler consisted of four 0.83 m sections that could be screwed together. One end of the sampler has cutter teeth and the other end has a removable thread protector that can be screwed onto the top section of the tube. When assembled, the sampler has graduations in units of 1 cm. Slits along the side of the tube allow the observer to determine the length of the core when it is in the tube. The spring scale that comes with the Federal Sampler is in units of cm SWE.  

To take a measurement with the Federal Sampler the following steps were taken:
\begin{enumerate}
\item Three depth measurements (within $\sim$50 cm of each other) were made using an avalanche probe and the depths were recorded.
\item The weight of the assembled empty tube was measured using the spring scale (units of cm SWE) and then recorded (tare).
\item The tube was placed vertically into the snow and then pushed on and twisted clockwise so that the cutters at the end of the tube would penetrate the snow pack. If this proved to be too difficult, the T-handle was added onto the tube to aid in pushing the tube further into the snow. 
\item When the bottom of the snow pack was reached (impenetrable ice layer), the observer would measure the snow depth by using the graduation on the outside of the tube.
\item The tube was then gently pulled out of the snow (so as not to lose any snow from the bottom). The length of the snow core inside the tube was then measured by using the side slits to see the top of the core and lining it up with the graduation on the outside of the tube and the recorded. 
\item The snow and tube were weighed together using the spring scale and the value was recorded.
\item The tube was then emptied and wiped using the ``dog bootie'' to remove any moisture. 
\end{enumerate}

\subsection{Firn Corer}


The firn corer was intended to be used in the accumulation area to extract a snow/firn core. The core would be used to determine the location of the snow/firn transition and from that, determine the snow depth and the mass of the snow core. 

When the corer was used in the field however, a number of problems were encountered that prevented the collection of relevant measurements. The first was that the snow core would get stuck inside the core barrel. Warm temperatures meant that the barrel would get wet from snow melt and when it was inserted into the snow pack, the snow would freeze onto the side of the barrel. This meant that the core could not be extracted as a single piece and removing the snow from the barrel was incredibly time consuming. The second problem, which was connected to the first problem, was that the coring chips in the hole could not be identified. Coring chips are loose snow crystals that fall to the bottom of the drilling hole when the barrel is taken out. When the barrel is reinserted for the second (or third) core, the snow in the subsequent core will contain these coring chips, which are not part of the intended core. The mass of the core will be incorrect (overestimate) because of this additional snow. This problem is typically avoided by extracting the core and identifying and removing the coring chips. However, since the core could not be extracted as one piece, this step was not possible and the masses of the cores was incorrect. 

In a few locations, sections of the core could be exacted without breaking them apart. In these areas the snow/firn transition could be identified. This means that in principal, the firn corer could be used to determine the depth of this transition. A few problems would need to be solved for this 
\begin{itemize}[noitemsep]
\item 
\end{itemize}


\subsection{Snowpit}
\label{sec:snowpit}

Three snowpits were dug on each glacier and the density was sampled every 10 cm using a wedge cutter. The snow temperature was also measured at 10 cm intervals. 

The procedure to complete measurements in the snowpit were as follows:
\begin{enumerate}
\item The face of the wall that was chosen for sampling was smoothed and a ruler was placed against the wall with the 0 cm mark at the bottom of the snowpit. The ruler was used to measure relevant heights within the snow pack. The snow surface above this wall was kept untouched during the digging process so that the true snow depth could be determined. 
\item Air and snow surface temperature were measured by placing the thermometer in the shade of a shovel or ski. 
\item A snow density sample was taken in 10 cm intervals through the full depth of the snow pack. Measurements were offset from each other so that the snow was not affected by previous measurements. 
	\begin{enumerate}
	\item The wedge cutter was inserted into the snow vertically (to sample 10 cm intervals) and the top was slid onto the wedge to isolate the sample. The wedge was taken out and inspected. If the sample appeared to fill the entire wedge (no obvious voids) then the wedge was emptied into a small plastic bag. If the sample was poor then the snow was discarded and a new sample was taken at the same height in the snow pit. 
	\item A spring scale (units of grams) was then used to weigh the bag with the snow sample and the weight was recorded. The snow sample was then discarded. Note that the spring scale was adjusted so that it read `0 g' with an empty bag.
	\end{enumerate}
\item Snow temperatures were also measured and recorded every 10 cm. The thermometer was inserted into the snow at the desired location and left to equilibrate. The temperature was then recorded.
\end{enumerate}

Modifications to this procedure occurred when snow samples could not be taken because the snow was too dense. This would often occur when ice layers or lenses were present in the snow, which could not be cut by the wedge cutter. In these cases, the thickness and approximate density (e.g. ice density of 917 kg/m$^3$) was recorded. A sample would then be taken using the wedge cutter but aligned horizontally so that a 5 cm tall sample was taken. Additionally, the sample interval closest to the ice surface (0--10 cm) would be difficult to obtain because the ice was rough and the snow above was faceted. Sometimes, this sample could not be obtained or a 5 cm sample needed to be taken. 













\end{document}