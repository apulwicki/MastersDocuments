%%%%%%%%%%%%%%%%%%%%%%%%%%%%%%%%%%%%%%%%%%%%
% 
% Last edits: Nov 9, 2016
%%%%%%%%%%%%%%%%%%%%%%%%%%%%%%%%%%%%%%%%%%%%

\documentclass[12pt]{article}
\usepackage{natbib}
\usepackage[letterpaper, margin=1in]{geometry}
\usepackage{graphicx}
\usepackage[table,xcdraw]{xcolor}
\usepackage{wrapfig}
\usepackage{enumitem}
\setlist[enumerate]{itemsep=0mm}
\usepackage{multirow}
\usepackage{lscape}
\usepackage{caption}
\usepackage{subcaption}
\usepackage{float}
\usepackage{hyperref}


\begin{document}
\noindent{Alexandra Pulwicki \\ \today}

\begin{center}
\Large \textbf{Results\\ Observations}
\end{center}


\section*{Overview}

This document shows an overview of the data that are available for the project. It provides a first look at the processed data and provides figures that visualize the data. Section 1 examines the density data and uncertainty and potential systematic errors associated with measurements made in the snowpits and using a Federal Sampler. Section 2 briefly looks at the snow depths measured at the study glaciers. Section 3 visualizes the collected zigzag data. Section 4 examines the various ways to estimate snow water equivalent (SWE) at each measurement location. It focuses on options for how to interpolate density measurements and then visualizes estimated SWE at the measurement locations. 


\tableofcontents
\pagebreak



%%%
\section{Density Estimates}
%%%

\subsection{Basic statistics}

The standard deviation of each type of density measurement is less than 10\% of the mean density (Table \ref{tab:density_stats}). For snowpit derived densities, the mean density is within one standard deviation between glaciers . The densities estimated using the Federal Sampler differed between glacier within one standard deviation. Our density measurements on Glacier 2 were lower than those on Glacier 4, while density measurements taken on Glacier 13 were the same as Both Glaciers 2 and 4. The mean of all Federal Sampler derived density values was skewed by the proportionally large number of measurements obtained on Glacier 13.

\begin{table}[h!]
\centering
\caption{Statistics of integrated densities measured using Federal Sampler or vertical density profiles (of snow wedge measurmenets) in snow pits. Mean, standard deviation (std), and number ($n$) of snow density (kg m$^{-3}$) measurements on study glaciers is shown.}
\label{tab:density_stats}
\begin{tabular}{c|ccc|ccc}
 & \multicolumn{3}{c}{\textbf{Snowpits}} & \multicolumn{3}{|c}{\textbf{Federal Sampler}} \\
\multirow{-2}{*}{\textbf{Glacier}} & Mean & Std & n & Mean & Std & n \\ \hline \hline
\textbf{Glacier 4} & 348 & 13 & 3 & 355 & 18 & 7 \\
\textbf{Glacier 2} & 333 & 26 & 4 & 286 & 34 & 7 \\
\textbf{Glacier 13} & 349 & 26 & 3 & 316 & 40 & 17 \\  \hline
\textbf{All} & 342 & 26 & 10 & 318 & 42 & 31
\end{tabular}
\end{table}

\subsection{Federal Sampler measurements and snow depth}

There is a positive linear relation (R$^2$ = 0.59, p$<$0.01) between measured snow density and depth for all Federal Sampler measurements (Figure \ref{fig:all_depth}). This positive relationship could be a result of physical processes, such as compaction, and/or artefacts during data collection; however, it seems more likely that this trend is a result of measurement artefacts for a number of reasons. First, the range of densities measured by the Federal sampler is large (225--410 kg m$^{-3}$) and the extreme values seem unlikely to exist at these study glaciers, which experience a continental snowpack with minimal mid-winter melt events. Previous unpublished density measurements taken on Glacier 2 range between 264 and 396 kg m$^{-3}$ for five study years with a maximum density difference of 110 kg m$^{-3}$ in any one year (Flowers, 2016, personal communication). Second, compaction effects would likely be small at these study glaciers because of the relatively shallow snowpack (deepest measurement was 340 cm). Third, no linear relationship exists between depth and snowpit-derived density (R$^2$ = 0.05) as can be seen in a plot of the depth-density relationship in snowpits in Figure \ref{fig:all_depth}. Together, these reasons lead us to conclude that the Federal Sampler measurements are biased. Linear detrending can correct the density data but it was decided to use uncorrected data for future analysis.



\begin{figure}[p]
	\centering
	\includegraphics[width =0.85\textwidth]{DepthDensity_SWEonly.png}\\
	\caption{Relationship between measured density and snow depth for all Federal Sampler and snowpit locations.}
	\label{fig:all_depth}
\end{figure}


\subsection{Density uncertainties}

\subsubsection{Snowpit densities}

Uncertainty in estimating density from snowpits stems from measurement errors and incorrect assignment of density to layers that could not be sampled (i.e. ice lenses and `hard' layers). To determine a possible range of snowpit derived integrated snow density values, the original data were used and three quantities were varied. Ice layer density was varied between 700 and 900 kg m$^{-3}$, ice layer thickness was varied by $\pm$1 cm of the observed thickness, and the density of layers identified as being too hard to sample (but not ice) was varied between 600 and 700 kg m$^{-3}$. 

The range of integrated density values is always less than 15\% of the reference density, with the largest ranges present on Glacier 2 (Table \ref{tab:density}). Density values for shallow pits that contained ice lenses were particularly sensitive to changes in density and ice lens thickness. 

\begin{table}[]
\centering
\caption{Summary of reference and range of integrated snow density calculated in snowpits . The assumed density values arose from taking a density of 917 kg m$^{-3}$ was applied to ice layers and a density of 600 kg m$^{-3}$ was applied to layers that were described as `hard' and were too difficult to sample. To determine the error in estimating integrated snow density, the values of ice density, ice thickness, and the `hard' layer density was varied between 700 and 917 kg m$^{-3}$, $\pm$ 1 cm, and 500 and 600 kg m$^{-3}$, respectively.}
\label{tab:density}
\resizebox{\textwidth}{!}{%
\begin{tabular}{lcccccc}
 &  & \multicolumn{4}{c}{\textbf{Density (kg m$^{-3}$)}} &  \\
\multirow{-2}{*}{} & \multirow{-2}{*}{\textbf{Depth (m)}} & \textit{Assumed value} & \textit{Minimum} & \textit{Maximum} & \textit{Range} & \multirow{-2}{*}{\textbf{\begin{tabular}[c]{@{}c@{}}Range as \% \\ of assumed value\end{tabular}}} \\ \hline \hline
G02\_LSP & 44 & 360.9 & 328.6 & 377.3 & 48.7 & 13.5 \\
G02\_Z4A\_SWE & 35 & 325.8 & 307.9 & 344.7 & 36.8 & 11.3 \\
G02\_USP & 119 & 344.0 & 327.1 & 361.9 & 34.8 & 10.1 \\ 
G02\_ASP & 170 & 300.2 & 298.6 & 303.1 & 4.5 & 1.5 \\  \hline
G04\_LSP & 190 & 350.9 & 343.2 & 359.1 & 15.9 & 4.5 \\
G04\_USP & 160 & 333.4 & 316.6 & 349.6 & 33.0 & 9.9 \\
G04\_ASP & 285 & 359.7 & 356.6 & 362.4 & 5.8 & 1.6 \\  \hline
G13\_LSP & 20 & 383.0 & 383.0 & 383.0 & 0 & 0 \\
G13\_USP & 100 & 355.4 & 345.6 & 366.9 & 21.3 & 6.0 \\
G13\_ASP & 145 & 307.8 & 306.4 & 308.2 & 1.8 & 0.6
\end{tabular}%
}
\end{table}

\subsubsection{Federal Sampler densities}

Mean Federal Sampler derived density has a larger range of values over the study glaciers when compared to snowpit densities (Table \ref{tab:density_TubeRange}). The percent range is also larger than snowpit densities for many of the measurement locations. 

\begin{table}[]
\centering
\caption{Range of densities estimated from Federal Sampler measuresments. The number ($n$) of good quality measurements, as well as the minimum, maximum, and mean density are shown. The density range given as a percent of the mean density is also shown.}
\label{tab:density_TubeRange}
\begin{tabular}{lccccc}
\multicolumn{1}{c}{\multirow{2}{*}{\textbf{Location}}} & \multirow{2}{*}{\textbf{$n$}} & \multicolumn{3}{c}{\textbf{Density (kg m$^{-3}$)}} & \multirow{2}{*}{\textbf{\begin{tabular}[c]{@{}c@{}}Range as \%\\ of mean (\%)\end{tabular}}} \\
\multicolumn{1}{c}{} &  & Mean & Minimum & Maximum &  \\ \hline  \hline
G04\_Z3A\_SWE & 3 & 334 & 309 & 358 & 14 \\
G04\_USP & 6 & 311 & 274 & 353 & 22 \\
G04\_Z2A\_SWE & 3 & 360 & 303 & 431 & 35 \\
G04\_LSP & 7 & 272 & 250 & 297 & 13 \\
G04\_Z5B\_SWE & 2 & 337 & 324 & 350 & 7 \\
G04\_Z5A\_SWE & 3 & 311 & 275 & 351 & 21 \\
G04\_Z5C\_SWE & 2 & 361 & 350 & 373 & 6 \\  \hline
G02\_Z5C\_SWE & 2 & 296 & 245 & 347 & 28 \\
G02\_USP & 7 & 294 & 232 & 353 & 34 \\
G02\_Z7A\_SWE & 3 & 326 & 304 & 349 & 12 \\
G02\_Z7B\_SWE & 2 & 336 & 320 & 351 & 9 \\
G02\_Z7C\_SWE & 3 & 351 & 338 & 365 & 7 \\
G02\_Z3B\_SWE & 3 & 349 & 341 & 353 & 3 \\
G02\_LSP\_SWE & 7 & 331 & 302 & 349 & 13 \\ \hline
G13\_ASP & 8 & 343 & 277 & 395 & 33 \\
G13\_651 & 3 & 329 & 318 & 345 & 7 \\
G13\_652 & 2 & 319 & 291 & 346 & 15 \\
G13\_654 & 3 & 298 & 266 & 318 & 14 \\
G13\_655 & 1 & 300 & 300 & 300 & 0 \\
G13\_656 & 3 & 279 & 227 & 315 & 24 \\
G13\_657 & 3 & 331 & 323 & 338 & 4 \\
G13\_658 & 2 & 343 & 333 & 354 & 6 \\
G13\_659 & 3 & 245 & 232 & 258 & 7 \\
G13\_Z7C\_SWE & 2 & 270 & 253 & 287 & 9 \\
G13\_USP & 6 & 294 & 247 & 359 & 31 \\
G13\_Z4C\_SWE & 4 & 342 & 334 & 350 & 5 \\
G13\_744 & 3 & 323 & 298 & 347 & 14 \\
G13\_Z3B\_SWE & 3 & 333 & 308 & 351 & 12 \\
G13\_Z4B\_SWE & 2 & 332 & 312 & 351 & 11 \\
G13\_Z5A\_SWE & 3 & 276 & 240 & 301 & 17 \\
G13\_Z5B\_SWE & 2 & 255 & 254 & 257 & 1
\end{tabular}
\end{table}

\subsection{Comparing density from snowpit and Federal Sampler measurements}

To compare snowpit-derived densities and Federal Sampler-derived densities, eight Federal Sampler measurements were taken around two snowpit locations on each study glacier. The results are shown in Figure \ref{fig:density_pitVStube}. The overall range of Federal Sampler-derived densities is larger than that of the snowpit-derived density values. Within the range of possible values (minimum and maximum densities), the density values are indistinguishable for all snowpit locations, except for the accumulation snowpit on Galcier 13 (`G13\_ASP').

\begin{figure}[H]
	\centering
	\includegraphics[width =0.95\textwidth]{SnowpitVsSWEtube_all.png}\\
	\caption{Comparison of density estimated using wedge cutters in a snow pit and Federal Sampler measurements for three study glaciers. Error bars are minimum and maximum values for each estimate as seen in Table \ref{tab:density} and \ref{tab:density_TubeRange}. A 1:1 reference line is also shown.}
	\label{fig:density_pitVStube}
\end{figure}


\subsection{Density and elevation}

A linear fit between density and elevation is often used to interpolate density values between measurement locations. A summary of linear fits of the snowpit-derived and Federal Sampler-derived densities can be seen in Table \ref{tab:elev_regress}. There seems to be no generalization of elevation regressions between study glaciers and even between sampling methods. Note that since Federal Sampler measurements, which have been shown to have a significant relationship between snow depth and estimated density, are likely to have skewed regressions because snow depth is significantly correlated with elevation (as seen in Figure \ref{fig:depth_elev}). 

A plot of snowpit-derived density versus elevation can be seen in Figure \ref{fig:elev_snowpit} and a plot of Federal Sampler-derived density versus elevation can be seen in Figure \ref{fig:elev_tube}.


\begin{table}[]
\centering
\caption{Summary of linear regressions between snowpit-derived density and elevation ($z$) as well as Federal Sampler-derived densities and elevation ($z$) for the study area.}
\label{tab:elev_regress}
\begin{tabular}{lcccc}
\multicolumn{1}{c}{} & \multicolumn{2}{c}{\textbf{\begin{tabular}[c]{@{}c@{}}Snowpit \\ Regression\end{tabular}}} & \multicolumn{2}{c}{\textbf{\begin{tabular}[c]{@{}c@{}}Fed. Sampler\\ Regression\end{tabular}}} \\
\multicolumn{1}{c}{\multirow{-2}{*}{\textbf{Location}}} & Equation & R$^2$ & Equation & R$^2$ \\ \hline  \hline
Glacier 4 & 0.03$z+$274 & 0.16 & -016$z+$714 & 0.53 \\
Glacier 2 & -0.14$z+$659 & 0.75 & 0.24$z-$282 & 0.72 \\
Glacier 13 & -0.19$z+$796 & 1.00 & 0.12$z+$33 & 0.21 \\ \hline
All & -0.12$z+$618 & 0.50 & -0.14$z+$659 & 0.75
\end{tabular}
\end{table}


\begin{figure}[H]
  \makebox[\textwidth][c]{\includegraphics[width=1.2\textwidth]{DepthElevation.png}}%
	\caption{Relationship between measured snow depth and elevation at all sampling locations.}
	\label{fig:depth_elev}
\end{figure}


\begin{figure}[H]
	\centering
	\includegraphics[width = 0.7\textwidth]{ElevationVsSnowpit_all.png}\\
	\caption{Relationship between snowpit-derived density and elevation for all study glaciers.}
	\label{fig:elev_snowpit}
\end{figure}


\begin{figure}[H]
	\centering
	\includegraphics[width = 0.6\textwidth]{ElevationVsSWEtube_all.png}\\
	\caption{Relationship between Federal Sampler-derived density and elevation for all study glaciers.}
	\label{fig:elev_tube}
\end{figure}


\pagebreak

%%%%%%%%%%%%%%%%%%%%%%%%%%%%%%%%%%%%%%
\section{Linear and curvilinear transect snow depth data}

\begin{figure}[H]
    \centering
    \begin{subfigure}[b]{0.48\textwidth}
        \includegraphics[width=\textwidth]{box_depth_wZZ.png}
        \caption{ }
        \label{fig:box_depth_wZZ}
    \end{subfigure}
    ~
    \begin{subfigure}[b]{0.48\textwidth}
        \includegraphics[width=\textwidth]{box_depth_noZZ.png}
        \caption{}
        \label{fig:box_depth_noZZ}
    \end{subfigure}

    \caption{Boxplots of snow depth measured on study glaciers. All snow depth values shown in (a) and snow depth values only from transects shown in (b). Red line indicates median, blue box shows first quantiles (25th and 75th percentiles), bars indicate minimum and maximum values (excluding outliers), and red crosses show outliers, which are defined as being outside of the range of 1.5 times the quartiles (approximately $\pm2.7\sigma$).}
    \label{fig:box_depth}
\end{figure}

A summary of the measured snow depth on all study glaciers is shown in Figure \ref{fig:box_depth} where \ref{fig:box_depth_wZZ} shows all snow depth measured (including zigzag values) and \ref{fig:box_depth_noZZ} shows snow depth values collected along curvilinear and linear transects. In both cases, Glacier 4 has the largest median and range of snow depth values, while Glacier 13 has the smallest. 

\pagebreak
%%%%%%%%%%%%%%%%%%%%%%%%%%%%%%%%%%%%%%
\section{Zigzag snow depth data}

A comparison of measured snow depth for each zigzag is shown in Figure \ref{fig:ZZ_boxplot}. The zigzags on Glacier 4 show minimal variability with a small range of values observed and few outliers. The mean depth is significantly larger at the upper most zigzag. Zigzags on Glacier 2 show more variability. The range on the mid zigzag is the largest of all the zigzags measured and the highest zigzag has many outliers. The zigzags on Glacier 13 do not vary considerably in range, although the lower zigzags show a large number of outliers which may be a results of these locations being close to a supraglacier meltwater channel. 

The depth measured at each zigzag is shown in Figures \ref{fig:ZZ_G04}, \ref{fig:ZZ_G02}, and \ref{fig:ZZ_G13}. There is considerable variability both between zigzags and within each zigzag. For example, in Figure \ref{fig:ZZ_G04}, `G04\_Z5B' has a more uniform snow depth than `G04\_Z3A', which has a large range in depth values. 
{
\begin{wrapfigure}{r}{1.1\textwidth} 
	\centering
	\includegraphics[width = 1.1\textwidth]{Zigzag_Boxplot.png}\\
	\caption{Boxplot of zigzag depths measured at each zigzag location.}
	\label{fig:ZZ_boxplot}
\end{wrapfigure}
}

\begin{landscape}
\begin{figure}
	\centering
	\includegraphics[width = 23 cm]{ZigzagDepth_G04.png}\\
	\caption{Plot of depth measured at zigzags on Glacier 4.}
	\label{fig:ZZ_G04}
\end{figure}

\begin{figure}
	\centering
	\includegraphics[width = 23 cm]{ZigzagDepth_G02.png}\\
	\caption{Plot of depth measured at zigzags on Glacier 2.}
	\label{fig:ZZ_G02}
\end{figure}
\end{landscape}

\begin{figure}[H] 
	\centering
	 \includegraphics[width=0.9\textwidth]{ZigzagDepth_G13.png}%
	\caption{Plot of depth measured at zigzags on Glacier 13.}
	\label{fig:ZZ_G13}
\end{figure}


%%%%%%%%%%%%%%%%%%%%%%%%%%%%%%%%%%%%%%
\section{Snow water equivalent (SWE)}

Snow water equivalent (SWE) estimated for each sampling location can be seen in Figures  \ref{fig:SWEmap_S1} to \ref{fig:SWEmap_F4}. For maps of SWE calculated for each density option see the Appendix. Generally, SWE is highest on Glacier 4 and lowest on Glacier 13. Glacier 4 also shows considerable SWE variability within the basin, with both high and low values seen along a single transect. Note that the individual measurement locations overlap on the figure.

\begin{figure}[H]
	\centering
	\includegraphics[width = \textwidth]{SWEmap_opt2.png}\\
	\caption{Estimated snow water equivalent (SWE) at measurement locations. Density was taken to be the mean value of all snowpit-derived densities (S1). Arrow shows ice flow direction.}
	\label{fig:SWEmap_S1}
\end{figure}

\begin{figure}[H]
	\centering
	\includegraphics[width = \textwidth]{SWEmap_opt3.png}\\
	\caption{Estimated snow water equivalent (SWE) at measurement locations. Density was taken to be the mean value of all snowpit-derived densities (F1). Arrow shows ice flow direction.}
	\label{fig:SWEmap_F1}
\end{figure}

\begin{figure}[H]
	\centering
	\includegraphics[width =\textwidth]{SWEmap_opt4.png}\\
	\caption{Estimated snow water equivalent (SWE) at measurement locations. Density was taken to be the mean value of snowpit-derived densities for each glacier (S2). Arrow shows ice flow direction.}
	\label{fig:SWEmap_S2}
\end{figure}

\begin{figure}[H]
	\centering
	\includegraphics[width = \textwidth]{SWEmap_opt5.png}\\
	\caption{Estimated snow water equivalent (SWE) at measurement locations. Density was taken to be the mean value of Federal Sampler-derived densities for each glacier (F2). Arrow shows ice flow direction.}
	\label{fig:SWEmap_F2}
\end{figure}

\begin{figure}[H]
	\centering
	\includegraphics[width = \textwidth]{SWEmap_opt6.png}\\
	\caption{Estimated snow water equivalent (SWE) at measurement locations. Density was determined by using a linear fit between snowpit-derived density and elevation for each glacier (S3). Arrow shows ice flow direction.}
	\label{fig:SWEmap_S3}
\end{figure}

\begin{figure}[H]
	\centering
	\includegraphics[width = \textwidth]{SWEmap_opt7.png}\\
	\caption{Estimated snow water equivalent (SWE) at measurement locations.Density was determined by using a linear fit between Federal Sampler-derived density and elevation for each glacier (F3). Arrow shows ice flow direction.}
	\label{fig:SWEmap_F3}
\end{figure}

\begin{figure}[H]
	\centering
	\includegraphics[width = \textwidth]{SWEmap_opt8.png}\\
	\caption{Estimated snow water equivalent (SWE) at measurement locations. Density was calculated using inverse distance weighting using all snowpit-derived densities (S4). Arrow shows ice flow direction.}
	\label{fig:SWEmap_S4}
\end{figure}

\begin{figure}[H]
	\centering
	\includegraphics[width =\textwidth]{SWEmap_opt9.png}\\
	\caption{Estimated snow water equivalent (SWE) at measurement locations. Density was calculated using inverse distance weighting using all snowpit-derived densities (F4). Arrow shows ice flow direction.}
	\label{fig:SWEmap_F4}
\end{figure}


\end{document}