

 \documentclass[twocolumn, letterpaper]{igs}

% but use this version when submitting your article:
% \documentclass[review,oneside, letterpaper]{igs}

 % \usepackage{igsnatbib}
  \usepackage{lmodern}
\usepackage{amsmath,amssymb,amsthm}
\usepackage{natbib} 
\usepackage{wrapfig}
\usepackage{enumitem}
\usepackage{multirow}
\usepackage{tabularx}
\usepackage{booktabs}
\usepackage{lscape}

\usepackage{caption, threeparttable}


\usepackage[pdftex]{graphicx}
    \usepackage{epstopdf}
\usepackage{adjustbox}



\newcolumntype{C}{>{\centering\arraybackslash}X}
\renewcommand{\tabularxcolumn}[1]{m{#1}}%


% check if we are compiling under latex or pdflatex
%  \ifx\pdftexversion\undefined
%    \usepackage[dvips]{graphicx}
%  \else
%    \usepackage[pdftex]{graphicx}
%    \usepackage{epstopdf}
%    \epstopdfsetup{suffix=}
%  \fi

\setlength{\pdfpageheight}{11in}
  
\begin{document}
%% TOPO DETAILS

\begin{landscape}
\begin{table*}
\leftskip=-5cm
\begin{threeparttable}
    \captionsetup{singlelinecheck=off, skip=4pt}
\caption{Description of topographic parameters used in the linear regression.}
\label{tab:TopoParams}
%\hspace{-5cm}
\begin{tabularx}{22cm}{XXXXX}

\midrule
\textbf{\begin{tabular}[c]{@{}l@{}}Topographic\\ parameter\end{tabular}} & \textbf{Definition} & \textbf{\begin{tabular}[c]{@{}l@{}}Calculation \\ method\end{tabular}} & \textbf{Notes} & \textbf{Source} \\ \midrule
\textbf{Elevation ($z$)} & Height above sea level & Values taken directly from DEM &  &  \\ \midrule
\textbf{Distance from centreline ($d_C$)} & Linear distance from user-defined glacier centreline & Minimum distance between the Easting and Northing of the northwest corner of each grid cell and a manually defined centreline &  &  \\ \midrule
\textbf{Slope ($m$)} & Angle between a plane tangential to the surface (gradient) and the horizontal & \texttt{r.slope.aspect} module in GRASS GIS software run through QGIS &  & \cite{Mitavsova1993, Hofierka2009, Olaya2009} \\ \midrule
\textbf{Aspect ($\alpha$)} & Dip direction of the slope & \texttt{r.slope.aspect} module in GRASS GIS software run through QGIS & $\sin(\alpha)$, a linear quantity describing a slope as north/south facing, is used in the regression & \cite{Mitavsova1993, Hofierka2009, Olaya2009} \\ \midrule
\textbf{Mean curvature ($\kappa$)} & Average of profile (direction of the surface gradient) and tangential (direction of the contour tangent) curvature & \texttt{r.slope.aspect} module in GRASS GIS software run through QGIS & ($+$) mean-concave terrain and ($-$) mean-convex terrain & \cite{Mitavsova1993, Hofierka2009, Olaya2009} \\ \midrule
\textbf{``Northness'' ($N$)} & A value of $-1$ represents a vertical, south facing slope, a value of $+1$ represents a vertical, north facing slope, and a flat surface yields 0 & Product of the cosine of aspect and sine of slope &  & \citep{Molotch2005} \\ \midrule
\textbf{Wind exposure/shelter parameter (Sx)} &  & Executable obtained from Adam Winstral that follows the procedure outlined in \cite{Winstral2002} & Calculation based on selecting a cell within a certain angle and distance from the cell of interest that has the greatest upward slope relative to the cell of interest & \citep{Winstral2002}

\end{tabularx}
\end{threeparttable}
\end{table*}
\end{landscape}


\end{document}