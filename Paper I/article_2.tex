%%%%%%%%%%%%%%%%%%%%%%%%%%%%%%%%%%%%%%%%%
% Journal Article
% LaTeX Template
% Version 1.4 (15/5/16)
%
% This template has been downloaded from:
% http://www.LaTeXTemplates.com
%
% Original author:
% Frits Wenneker (http://www.howtotex.com) with extensive modifications by
% Vel (vel@LaTeXTemplates.com)
%
% License:
% CC BY-NC-SA 3.0 (http://creativecommons.org/licenses/by-nc-sa/3.0/)
%
%%%%%%%%%%%%%%%%%%%%%%%%%%%%%%%%%%%%%%%%%

%----------------------------------------------------------------------------------------
%	PACKAGES AND OTHER DOCUMENT CONFIGURATIONS
%----------------------------------------------------------------------------------------

\documentclass[twoside,twocolumn]{article}

\usepackage{blindtext} % Package to generate dummy text throughout this template 
\usepackage{natbib}

\usepackage[sc]{mathpazo} % Use the Palatino font
\usepackage[T1]{fontenc} % Use 8-bit encoding that has 256 glyphs
\linespread{1.05} % Line spacing - Palatino needs more space between lines
\usepackage{microtype} % Slightly tweak font spacing for aesthetics

\usepackage[english]{babel} % Language hyphenation and typographical rules

\usepackage[hmarginratio=1:1,top=32mm,columnsep=20pt]{geometry} % Document margins
\usepackage[hang, small,labelfont=bf,up,textfont=it,up]{caption} % Custom captions under/above floats in tables or figures
\usepackage{booktabs} % Horizontal rules in tables

\usepackage{lettrine} % The lettrine is the first enlarged letter at the beginning of the text

\usepackage{enumitem} % Customized lists
\setlist[itemize]{noitemsep} % Make itemize lists more compact

\usepackage{abstract} % Allows abstract customization
\renewcommand{\abstractnamefont}{\normalfont\bfseries} % Set the "Abstract" text to bold
\renewcommand{\abstracttextfont}{\normalfont\small\itshape} % Set the abstract itself to small italic text

\usepackage{titlesec} % Allows customization of titles
\renewcommand\thesection{\Roman{section}} % Roman numerals for the sections
\renewcommand\thesubsection{\roman{subsection}} % roman numerals for subsections
\titleformat{\section}[block]{\large\scshape\centering}{\thesection.}{1em}{} % Change the look of the section titles
\titleformat{\subsection}[block]{\large}{\thesubsection.}{1em}{} % Change the look of the section titles

\usepackage{fancyhdr} % Headers and footers
\pagestyle{fancy} % All pages have headers and footers
\fancyhead{} % Blank out the default header
\fancyfoot{} % Blank out the default footer
\fancyhead[C]{Running title $\bullet$ May 2016 $\bullet$ Vol. XXI, No. 1} % Custom header text
\fancyfoot[RO,LE]{\thepage} % Custom footer text

\usepackage{titling} % Customizing the title section

\usepackage{hyperref} % For hyperlinks in the PDF

%----------------------------------------------------------------------------------------
%	TITLE SECTION
%----------------------------------------------------------------------------------------

%\setlength{\droptitle}{-4\baselineskip} % Move the title up

%\pretitle{\begin{center}\Huge\bfseries} % Article title formatting
%\posttitle{\end{center}} % Article title closing formatting
%\title{Article Title} % Article title
%\author{%
%\textsc{John Smith}\thanks{A thank you or further information} \\[1ex] % Your name
%\normalsize University of California \\ % Your institution
%\normalsize \href{mailto:john@smith.com}{john@smith.com} % Your email address
%\and % Uncomment if 2 authors are required, duplicate these 4 lines if more
%\textsc{Jane Smith}\thanks{Corresponding author} \\[1ex] % Second author's name
%\normalsize University of Utah \\ % Second author's institution
%\normalsize \href{mailto:jane@smith.com}{jane@smith.com} % Second author's email address
%}
%\date{\today} % Leave empty to omit a date
%\renewcommand{\maketitlehookd}{%
%\begin{abstract}
%\noindent \blindtext % Dummy abstract text - replace \blindtext with your abstract text
%\end{abstract}
%}

%----------------------------------------------------------------------------------------

\begin{document}

% Print the title
%\maketitle

%----------------------------------------------------------------------------------------
%	ARTICLE CONTENTS
%----------------------------------------------------------------------------------------

\section{Introduction}

Objective: (1) Discuss choices made when moving from measurement to accumulation and (2) show how system variability and our choices interact to create uncertainty in our estimate of accumulation

- snow distribution in alpine regions is not uniform or static, but
rather highly variable and influenced by diverse and dynamic processes operating on multiple spatial and temporal scales -> topographic effects (crevasses, surface topo, elevation aspect, precip grad across range), snow drift and preferential deposition
-  [22] note that studies of snow water equivalent (SWE) that have been conducted in
alpine environments vary considerably in the extent and spacing of their measurements.
- Snow accumulation is spatially variable on point scales (<5 m), hillslope scales (1–100 m),
basin scales (100–10,000 m) and regional scales (10–1000 km) [22].
-Point-scale variability is generally associated with surface roughness effects and the
presence of small obstacles. -> take three measures
Many parts of a glacier though
are characterized by a relatively smooth surface, with roughness lengths on the order of
centimeters [57]. In these areas, point-scale variability of snow depth is low. However, in
heavily crevassed regions, point-scale variability can be large and thus exert a dominant
control on snow distribution in the area [82].
-Hillslope-scale variability is caused by variations in the surface topography of the glacier.
The curvature and slope of the surface as well as the presence of local ridges or depressions
can affect where snow is located [15, 115]. Avalanching can also redistribute snow, especially
on the margins of a glacier [17, 89].
Watershed-scale variability results mainly from the effects of changing elevation and
aspect on atmopsheric conditions [22]. In particular, orographic lifting and shading can
result in higher elevation and north-facing areas of the glacier having more snow than other
areas [89, 115]. Gradients in temperature from elevation changes also affect the freezing
level, which determines whether precipitation falls as snow or rain [17]. For example, [77]
found a strong influence of elevation in determining accumulation on Findel Glacier in
Switzerland.
Regional variability occurs when areas within a mountain range have differing amounts of
snow. Often, this results from horizontal precipitation gradients and rain shadows forming
on the lee side of topographic divides. Areas with large, steep mountains are especially
affected by these processes.

----------------------------------------------------

derived accumulation
estimated winter surface mass balance
distributed snow water equivalent
%------------------------------------------------

\pagebreak
\section{Methods}

study area overview

\subsection{Field Methods}

Snow depth ($d$) and density ($\rho$) measurements are needed to estimate accumulation (SWE $= d \times \rho$). Snow depth is generally accepted to be more variable than density  [38, 22] (cf??) so sampling designs are chosen to capture depth variability at multiple spatial scales and to account for known variation. Sampling designs need to avoids bias, allow for the greatest variability to be measured, and minimize distance travelled [110].

Snow depth was measured at three glaciers, along linear and curvilinear transects, and with multiple measurements at each location to account for range-, basin-, and point-scale variability, respectively. The precipitation gradient in the St. Elias Mounatins, Yukon (Taylor-Barge, 1969) is sampled by selecting Glaciers 4, 2, and 13 (naming adopted from ??), which are located increasing far from the head of the Kaskawalsh Glacier. Centreline and transverse transects, with sample spacing of $10-60$ m, are selected to capture established correlation between elevation and accumulation as well as accumulation differences between ice-marginal and center accumulation [125]. An hourglass and circle design, which allows for sampling in all directions and is easy to travel (Parr, C., 2016 personal communication), was also implemented. At each measurement location, we took $3-4$ depth measurements, resulting in more than 9,000 snow depth measurements throughout the study area. 

Our sampling campaign involved four people and occurred between May 5 and 15, 2015.
Measurement locations were predefined with a 60 m spacing along transects. While roped-up for glacier travel, the lead person used a handheld GPS (Garmin GPSMAP 64s) to navigate as close to the predefined locations as possible. The remaining three people used aluminium avalanche probes (3.2 m) to take $3-4$ snow depth measurements within $\sim$1 m of each other. Each observer was approximately 10 m behind the person ahead of them along the transect line. The location of each depth measurement was approximated based on the recorded location of the first person. 

When estimating accumulation, snow depth variability at scales less than the grid-size of satellite derived elevation models is classified as being caused by random effects that are assumed to be unbiased and unpredictable [127]. A linear-random sampling design, termed `zigzag', was implemented to capture grid-scale variability [110]. We measured depth at random intervals ($0.3 - 3.0$ m) along two `Z'-shaped transects within $40\times40$ m squared aligned with a randomly selected DEM grid cell.

Snow density was measured using a wedge cutter in three snowpits on each glacier. We collected a continuous density profile by inserting a $5\times5\times 10$ cm (250 cm$^3$) wedge-shaped cutter in 5 cm increments to extract snow samples and the weighted the sampled with a spring scale \citep{Gray1981,Firez2009}. While snow pits provide the most accurate measure of snow density, digging and sampling a snow pit is time and labour intensive. Therefore, a Federal Snow Sampler (FS) \citep{Clyde1932}, which measures bulk SWE, was used to augment the spatial extent of density measurements. Three measurements were taken at $7-19$ locations on each glacier and eight FS measurements were co-located with each snow pit profile.

\subsection{Analysis Methods}

Estimating accumulation from measured values of snow depth and density requires a number of processing steps. First, measured density is interpolated to estimate SWE at each depth sampling location. We chose four separate methods to interpolate density: (1) mean density for entire range, (2) mean density for each glacier, (3) density linear regression with elevation and (4) inverse-distance weighted density. Snow pit and Federal Sampler measures were used independently for each method -- due to poor correlation between data sets (see Results) -- so eight density interpolation options exist in this study. Second, we average data within a SPOT-5 DEM-aligned grid ($40\times40$ km) (Korona et al., 2009). 

(Equations- LR, MLR, BIC, BMA, kriging??)

Data are then interpolated using linear regression (LR), simple kriging (SK), as well as regression kriging (RK). Linear regressions relate observed SWE to a linear combination of DEM-derived topographic parameters. Topographic parameters are weighted by a set of fitted regression coefficients ($\beta_i$) and then summed and area-averaged to estimate the mean specific winter balance ([m w.e.]). To prevent data over fitting, cross validation is implemented --- 1000 random subsets (2/3 points) of the data are used to fit the LR while the RMSE between the unused data (1/3 points) and predicted data is calculated. Regression coefficients resulting in the lowest RMSE are selected. A LR is calculated for all possible linear combinations of topographic parameters, which are then weighted according to their relative predictive success - as assessed by the Bayesian information criterion (BIC) value - to obtain the final regression coefficients. We used multiple linear regression (MLR) as well as Bayesian model averaging (BMA), two types of fitting algorithms, to estimate initial $\beta$ values. MLR estimates regression coefficients by minimizing the sum of squares of the vertical deviations of each data point from the regression line. BMA, implemented using the BMS R package (Zeugner and Feldkircher, 2015), uses Bayes' theorem to find the probability distribution of $\beta$ values and then weights all linear combinations of topographic parameters by the calculated model probability. SWE distribution was then estimated by applying the regression coefficients to the value of each topographic parameter in the DEM grid cells. 

Topographic parameters were derived from a SPOT-5 DEM ($40\times40$ km) (Korona et al., 2009). Elevation ($z$) values were taken from the SPOT-5 DEM directly. Distance from centreline ($d_C$) was calculated as the minimum distance between the Easting
and Northing of the northwest corner of each grid cell and a manually defined centreline. Slope ($m$) is defined as the angle between a plane tangential to the surface (gradient) and the horizontal [94]. Aspect ($\alpha$) represents the orientation of the steepest slope and sine of aspect (North/South facing slope) is used. Mean curvature ($\kappa$) is found by taking the average of profile and tangential curvature and emphasizes mean-concave (positive values) areas with relative accumulation and mean-convex (negative values) terrain with relative scouring [94]. Slope, aspect, and curvature were calculated using the \texttt{r.slope.aspect} module in GRASS GIS software run through QGIS as described in [84] and [58].  ``Northness'' ($N$) is defined as the product of the cosine of aspect and sine of slope [86]. A value of -1 represents a vertical, south facing slope, a value of +1 represents a vertical, north facing slope, and a flat surface yields 0. Sx represents wind exposure/shelter and is based on selecting a cell within a certain angle and distance from the cell of interest that has the greatest upward slope relative to the cell of interest [133]. Sx was determined
using a executable obtained from Adam Winstral that follows the procedure outlined in
[133]. The ranges of all topographic parameters covered by the measurements represent more than 70\% of the total area of each glacier (except for the elevation range on Glacier 2, which was 50\%).

Visual inspection of the curvature fields calculated using the DEM showed noisy spatial
distribution that did not vary smoothly. To minimize the effect of noise on parameters sensitive to DEM grid cell size, we applied a $7\times7$ smoothing window to the DEM, which was then used to calculate curvature, slope, aspect and ``northness''.

Simple kriging (SK) estimates SWE values at unsampled locations by using the isotropic spatial correlation (covariance) of measured SWE to find a set of optimal weights. SK assumes that if sampling points are distributed throughout a surface, the degree of spatial correlation of the observed surface can be determined and the surface can then be interpolated between sampling points. We used the DiceKriging R package (Roustant et al., 2012) to calculate the maximum likelihood covariance matrix, as well as range distance (measure of correlation length) and nugget (residual that encompasses sampling-error variance as well as the spatial variance at distances less than the minimum sample spacing). 

The regression kriging (RK) estimate was found by first calculating the residuals from the LR estimate at measurement locations. Then, distributed residuals were estimated using SK, and the linear regression SWE and kriged residuals were added to obtain a RK estimate of distributed SWE. Regression kriging can be thought of as an intermediate between pure kriging (no regression) and pure regression (small residuals) and can be more strongly skewed to either end-member based on the strength of the regression correlation [55]

%------------------------------------------------

%\section{Results}

%\begin{table}
%\caption{Example table}
%\centering
%\begin{tabular}{llr}
%\toprule
%\multicolumn{2}{c}{Name} \\
%\cmidrule(r){1-2}
%First name & Last Name & Grade \\
%\midrule
%John & Doe & $7.5$ \\
%Richard & Miles & $2$ \\
%\bottomrule
%\end{tabular}
%\end{table}

%\blindtext % Dummy text

%\begin{equation}
%\label{eq:emc}
%e = mc^2
%\end{equation}

%\blindtext % Dummy text

%------------------------------------------------
\pagebreak
\pagebreak
\section{Discussion}
[77] conducted an airborne GRP survey of two adjacent glaciers in Switzerland. The
lower part of the larger valley glacier showed a clear correlation between altitude and snow
accumulation. The upper part of the glacier and the adjacent smaller glacier had no alti-
tudinal trend and the fluctuations in depth were large. Additionally, the accumulation was
40\% lower on the smaller glacier. The altitudinal trend is a well documented pattern and
was thought to be a result of melt that occurred during warmer weather, which is more
pronounced at lower elevations. Spatial variability of precipitation and redistribution of
snow were believed to have resulted in the high spatial variability in higher parts of the
study area. Since the majority of the precipitation events originated from one direction and
the large glacier was on the lee side of a ridge, it experienced preferential deposition. Mean-
while, the smaller glacier was further along the storm track so it received less precipitation.
Overall, [77] showed that snow distribution on glaciers is not simply a function of altitude,
which corroborated research done in other alpine catchments.

 In most cases, the resolution of measurements over a large area is insufficient to
approximate the true variability [15, 32].
%----------------------------------------------------------------------------------------
%	REFERENCE LIST
%----------------------------------------------------------------------------------------

\bibliography{/home/glaciology1/Documents/MastersDocuments/MastersLit}
\bibliographystyle{igs}

%----------------------------------------------------------------------------------------

\end{document}
